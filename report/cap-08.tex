\chapter{Conclusão}
\label{chapter:conclusao}
\noindent

Como esperado, as recomendações do \textit{Google} no artigo de referência se mantiveram coerentes com os resultados atingidos. O melhor resultado com a métrica de acurácia foi o \textit{Stacked RNN}, que com o mesmo número de dimensões, obteve um resultado de 87,37\%, comparado aos 85,6\% do RNN e aos 84,4\% do SepCNN. Porém, ao se observar o tempo de treinamento dos modelos, percebe-se que, utilizando o mesmo número de \textit{epochs}, o SepCNN tem o melhor custo/benefício, terminando seu treinamento com apenas 12 minutos e 8 segundos, tempo consideravelmente menor do que os modelos que utilizam RNN, que, executados nas mesmas condições, levaram pelo menos de 3 horas e 30 minutos para terminarem seus treinamentos. Os valores de acurácia e tempo dos classificadores estão descrito na tabela \ref{tab:comparisons}, sendo que os modelos de \textit{word embeddings} listados utilizam palavras com 50 dimensões.

\begin{table}[ht]
\centering
\caption{Comparação entre os classificadores}
\vspace{0.5cm}
\begin{tabular}{c|c|c}
 
Classificador & Acurácia & Tempo de treinamento \\
\hline
NB & 0.79287 & -//- \\
SVM & 0.77368 & -//- \\
NN Simples & 0.7377 & \textbf{04m02s} \\
SepCNN & 0.8444& 12m08s \\
RNN  & 0.8560 & 03h31m03s \\
\textit{Stacked}RNN & \textbf{0.8737} & 06h23m15s
\end{tabular}
\label{tab:comparisons}
\end{table}

Outro produto relevante deste trabalho foi a ferramenta de aplicação de modelos de classificação que se mostra modular o suficiente para ser aplicada a diversos contextos distintos, não se restringindo às questões utilizadas neste artigo. 

Dentre os possíveis próximos trabalhos, encontram-se a continuação da ferramenta de testes de modelos utilizando integração contínua e automação de infraestrutura para tornar o processo de testes mais rápido, de modo que novos modelos poderão ser testados sem preocupação de configuração de máquinas, e que possuam um custo menor. Este tipo de implementação também possibilitaria a expansão do número de dimensões dos modelos, podendo chegar ao uso de 1000 dimensões.

Para que a classificação tenha uso prático, é necessário que ela seja capaz de distinguir além dos temas das questões, seus subtemas, de modo a categorizar ainda mais o banco de questões. Além disso, a expansão dos modelos para outros tipos de questões diferentes do conjunto de treinamento como, por exemplo, questões de vestibulares ou exames de periódicos de escolas e este caso fica como uma sugestão de trabalho futuro.