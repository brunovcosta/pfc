%%
%
% ARQUIVO: pre-texto.tex
%
% VERSÃO: 1.0
% DATA: Maio de 2016
% AUTOR: Coordenação de Trabalhos Especiais SE/8
% 
%  Arquivo tex para a criação da parte pré-textual do documento de Projeto de Fim de Curso.
%
%%


% -----
% PÁGINA DE CAPA DO DOCUMENTO DE PFC
% -----
\makecapa

% -----
% PÁGINA DE TÍTULO DO PFC
% -----
\prepareadvisors
\maketitle

% -----
% PÁGINA DE CRÉDITOS DO DOCUMENTO DE PFC
% -----
\makecredits

% -----
% PÁGINA DE FOLHA DE ASSINATURAS
% -----
\preparemembers
\approvalpage

% -----
% PÁGINA DE DEDICATÓRIA (OPCIONAL, ie. pode remover toda a página)
% -----
%%% DEDICATÓRIA - PREENCHER...
\dedicatoria{%
Ao Instituto Militar de Engenharia, alicerce da minha formação e aperfeiçoamento.
}%
\makededication

% -----
% PÁGINA DE AGRADECIMENTOS (OPCIONAL, ie. pode remover toda a página)
% -----
%%% AGRADECIMENTOS - PREENCHER...
\agradecimentos{%
Agradeço a todas as pessoas que me incentivaram, apoiaram e possibilitaram esta oportunidade de ampliar meus horizontes. \\
\indent
Meus familiares, cônjuge e mestres.\\
\indent
Em especial ao meu Professor Orientador Dr. Alex Garcia.
}%
\makethanks

% -----
% PÁGINA DE EPÍGRAFE (OPCIONAL, ie. pode remover toda a página)
% -----
%%% EPÍGRAFE - PREENCHER...
\epigrafe{%
Not everything that can be counted counts; and not everything that counts can be counted.}%
\autorepigrafe{%    %% Se não tem autor, coloque "Anônimo"
Anônimo
}%
\makeepigraph

% -----
% PÁGINA DE SUMÁRIO
% -----
\tableofcontents

% -----
% PÁGINAS DE LISTAS DE FIGURAS E DE TABELAS
% se a Dissertação não possui figuras e/ou tabelas, REMOVA O COMANDO CORRESPONDENTE
% -----
%\listoffigures
%\listoftables

% -----
% PÁGINA DE LISTA DE SIGLAS
% se a Dissertação não possui siglas, REMOVA TODA A PÁGINA
% -----
%%% SIGLAS - PREENCHER...
\acronimo{EAD}{Ensino à distância}
\acronimo{PAA}{Plataforma de Auxílio ao Aprendizado}
\acronimo{IA}{Inteligência Artificial}
\acronimo{CAA}{Classificação Automática de Assuntos}
\acronimo{OCR}{Optical Character Recognition}
\acronimo{NN}{Neural Network}
\acronimo{CNN}{Convolutional Neural Network}
\acronimo{RNN}{Recurrent Neural Network}

\listofnicks

% -----
% PÁGINA DE LISTA DE ABREVIATURAS
% se a Dissertação não possui abreviaturas ou símbolos, REMOVA TODA A PÁGINA
% -----
%%% ABREVIATURAS - PREENCHER...
%\abreviatura{Ja}{jacobiano}
%\abreviatura{JS}{fluxo secundário (difusivo)}
%\abreviatura{M}{número de Mach}
%
%%%% SÍMBOLOS - PREENCHER...
%\simbolo{$\Phi$}{termo de dissipação viscosa}
%\simbolo{$\Gamma$}{coeficiente de difusão efetivo}
%\simbolo{$\alpha$}{fator de sub-relaxação}
%\simbolo{$\phi$}{variável dependente da equação diferencial geral}

%\listofsymbols

% -----
% PÁGINA DE RESUMO
% -----
%%% RESUMO - PREENCHER...
\resumo{%
Ferramentas de auxílio ao aprendizado são usadas nas mais diversas etapas da educação formal ou informal. Estas ferramentas costumam empregar diversas tecnologias. Dentre elas, pode-se incluir a classificação de documentos, pois essas plataformas são, em geral, grandes agregadoras de conteúdo de ensino. Para uma melhor acessibilidade de todo esse conteúdo, é necessário que ele seja categorizado, o que nem sempre é viável de ser realizado manualmente devido ao grande volume. Nesse contexto, entra a Classificação automática documentos.\\
\indent
Este trabalho tem como objetivo analisar e implementar modelos consagrados de classificação de texto, aplicando-os para a classificação de enunciados de questões em avaliações de ensino segundo seus assuntos. A partir da exploração dessas técnicas, desenvolver-se-á um modelo híbrido de classificador especializado em enunciados de questões.}%
\makeresumo

% -----
% PÁGINA DE ABSTRACT
% -----
%%% ABSTRACT - PREENCHER...
\abstract{%
Learning aid tools are used in the many stages of formal or informal education. These tools usually employ various technologies. Among them, one can include the classification of documents, since these platforms are, in general, great aggregators of teaching content. For better accessibility of all this content, it must be categorized, which is not always feasible to be performed manually due to the large volume. In this context, enter the Automatic Classification documents.\\
\indent
This work aims to analyze and implement established text classification models, applying them to the classification of question statements in teaching assessments according to their subjects. From the exploration of these techniques, a hybrid classifier model will be developed that specializes in question statements.
}%
\makeabstract