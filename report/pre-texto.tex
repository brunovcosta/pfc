%%
%
% ARQUIVO: pre-texto.tex
%
% VERSÃO: 1.0
% DATA: Maio de 2016
% AUTOR: Coordenação de Trabalhos Especiais SE/8
% 
%  Arquivo tex para a criação da parte pré-textual do documento de Projeto de Fim de Curso.
%
%%


% -----
% PÁGINA DE CAPA DO DOCUMENTO DE PFC
% -----
\makecapa

% -----
% PÁGINA DE TÍTULO DO PFC
% -----
\prepareadvisors
\maketitle

% -----
% PÁGINA DE CRÉDITOS DO DOCUMENTO DE PFC
% -----
\makecredits

% -----
% PÁGINA DE FOLHA DE ASSINATURAS
% -----
\preparemembers
\approvalpage

% -----
% PÁGINA DE DEDICATÓRIA (OPCIONAL, ie. pode remover toda a página)
% -----
%%% DEDICATÓRIA - PREENCHER...
% \dedicatoria{%
% Ao Instituto Militar de Engenharia, alicerce da minha formação e aperfeiçoamento.
% }%
% \makededication

% -----
% PÁGINA DE AGRADECIMENTOS (OPCIONAL, ie. pode remover toda a página)
% -----
%%% AGRADECIMENTOS - PREENCHER...
\agradecimentos{%
Agradecemos a todas as pessoas que nos incentivaram, apoiaram e possibilitaram esta oportunidade de ampliar nossos horizontes. \\
\indent
Em especial ao Professor Orientador Dr. Alex Garcia.
}%
\makethanks

% -----
% PÁGINA DE EPÍGRAFE (OPCIONAL, ie. pode remover toda a página)
% -----
%%% EPÍGRAFE - PREENCHER...
\epigrafe{%
Not everything that can be counted counts; and not everything that counts can be counted.}%
\autorepigrafe{%    %% Se não tem autor, coloque "Anônimo"
Anônimo
}%
\makeepigraph

% -----
% PÁGINA DE SUMÁRIO
% -----
\tableofcontents

% -----
% PÁGINAS DE LISTAS DE FIGURAS E DE TABELAS
% se a Dissertação não possui figuras e/ou tabelas, REMOVA O COMANDO CORRESPONDENTE
% -----
\listoffigures
\listoftables

% -----
% PÁGINA DE LISTA DE SIGLAS
% se a Dissertação não possui siglas, REMOVA TODA A PÁGINA
% -----
%%% SIGLAS - PREENCHER...
\acronimo{API}{\textit{Application programming interface}}
\acronimo{AWS}{\textit{Amazon Web Services}}
\acronimo{CNN}{\textit{Convolutional Neural Network}}
\acronimo{CPU}{\textit{Central processing unit}}
\acronimo{EAD}{Ensino à distância}
\acronimo{GPU}{\textit{Graphics Processing Unit }}
\acronimo{HTML}{\textit{Hypertext Markup Language}}
\acronimo{HTTP}{\textit{Hypertext Transfer Protocol}}
\acronimo{IA}{Inteligência Artificial}
\acronimo{JSON}{\textit{JavaScript Object Notation}}
\acronimo{LSTM}{\textit{Long Short Term Memory}}
\acronimo{NB}{\textit{Naive Bayes}}
\acronimo{NN}{\textit{Neural Network}}
\acronimo{RNN}{\textit{Recurrent Neural Network}}
\acronimo{SepCNN}{\textit{Separable Depthwise Convolutional Neural Network}}
\acronimo{SVM}{\textit{Support Vector Machine}}
\acronimo{TF-IDF}{\textit{Term Frequency-Inverse Document Frequency}}
\acronimo{VM}{\textit{Virtual Machine}}
\listofnicks

% -----
% PÁGINA DE LISTA DE ABREVIATURAS
% se a Dissertação não possui abreviaturas ou símbolos, REMOVA TODA A PÁGINA
% -----
%%% ABREVIATURAS - PREENCHER...
%\abreviatura{Ja}{jacobiano}
%\abreviatura{JS}{fluxo secundário (difusivo)}
%\abreviatura{M}{número de Mach}
%
%%%% SÍMBOLOS - PREENCHER...
%\simbolo{$\Phi$}{termo de dissipação viscosa}
%\simbolo{$\Gamma$}{coeficiente de difusão efetivo}
%\simbolo{$\alpha$}{fator de sub-relaxação}
%\simbolo{$\phi$}{variável dependente da equação diferencial geral}

%\listofsymbols

% -----
% PÁGINA DE RESUMO
% -----
%%% RESUMO - PREENCHER...
\resumo{%
Ferramentas de auxílio ao aprendizado são usadas nas mais diversas etapas da educação formal ou informal. Estas ferramentas costumam empregar diversas tecnologias. Dentre elas, pode-se incluir a classificação de documentos, pois essas plataformas são, em geral, grandes agregadoras de conteúdo de ensino. Para uma melhor acessibilidade de todo esse conteúdo, é necessário que ele seja categorizado, o que nem sempre é viável de ser realizado manualmente devido ao grande volume e a necessidade de conhecimento técnico sobre os assuntos abordados. Nesse contexto, pode ser aplicada a Classificação Automática de Documentos.\\
\indent
Este trabalho tem como objetivo analisar e implementar modelos consagrados de classificação de texto, aplicando-os para a classificação de enunciados de questões em avaliações de ensino segundo seus assuntos. A partir da exploração dessas técnicas, desenvolveu-se um modelo híbrido de classificador especializado em enunciados de questões.}%
\makeresumo

% -----
% PÁGINA DE ABSTRACT
% -----
%%% ABSTRACT - PREENCHER...
\abstract{%
Learning aid tools are used in the most diverse stages of formal or informal education. These tools usually employ various technologies. Among them, one can include the classification of documents, since these platforms are, in general, great aggregators of teaching content. For a better accessibility of all this content, it needs to be categorized, which is not always feasible to be done manually due to the large volume and the need for technical knowledge on the issues addressed. In this context, it's interesting to apply the Automatic Document Classification.\\
\indent
This work aims to analyze and implement established text classification models, applying them to the classification of question statements in teaching assessments according to their subjects. From the exploration of these techniques, a hybrid classifier model was developed that specializes in question statements.
}%
\makeabstract