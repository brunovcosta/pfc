%%
%
% ARQUIVO: cap-01.tex
%
% VERSÃO: 1.0
% DATA: Maio de 2016
% AUTOR: Coordenação de Trabalhos Especiais SE/8
% 
%  Arquivo tex de exemplo de capítulo do documento de Projeto de Fim de Curso.
%
% ---
% DETALHES
%  a. todo capítulo deve começar com \chapter{•}
%  b. usar comando \noindent logo após \chapter{•}
%  c. citações para referências podem ser
%       i. \citet{•} para citações diretas (p. ex. 'Segundo Autor (2015)...'
%       ii. \citep{•} para citações indiretas (p. ex. '... (AUTOR, 2015)...'
%  d. notas de rodapé devem usar dois comandos
%       i. \footnotemark para indicar a marca da nota no texto
%       ii. \footnotetext{•}, na sequência, para indicar o texto da nota de rodapé
%  e. figuras devem seguir o exemplo
%       i. devem ficar no diretório /img e devem ser no formato EPS
%  f. tabelas devem seguir o exemplo
%  g. figuras e tabelas podem ser colocadas em orientação landscape
%       i. figuras: usar \begin{sidewaysfigure} ... \end{sidewaysfigure}
%                   em vez de \begin{figure} ... \end{figure}
%       ii. tabelas: usar \begin{sidewaystable} ... \end{sidewaystable}
%                    em vez de \begin{table} ... \end{table}
%  h. toda figura e tabela deve ser referenciada ao longo do texto com \ref{•}
% ---
%%

\chapter{Introdução}
\noindent

\section{Motivação}
Questões são parte fundamental do processo de aprendizado e sua classificação por assunto e dificuldade é de suma importância nas mais variadas aplicações educacionais.

Por meio dessa classificação, professores podem elaborar testes de maneira assistida e mais eficiente.

Outra aplicação da classificação de questões trata das ferramentas adaptativas. Nesse tipo de ferramenta, estudantes podem executar baterias de exercícios em série cujo conteúdo varia de acordo com o desempenho de acerto otimizando o aprendizado.

Questões são muito usadas como forma de avaliação de conhecimento em processos seletivos e sua maioria consiste de textos e imagens justapostos divididos entre enunciado e alternativas. A recorrência na forma permite que métodos específicos de classificação tenham mais eficiência que abordagens em texto genérico.

\section{Objetivo}
Este trabalho tem como objetivo a implementação, teste e comparação das principais técnicas de classificação de texto aplicando-as em questões de concursos públicos.

Para fim de simplificação de escopo, serão consideradas apenas questões objetivas, que contenham alternativas, e retiradas de uma mesma fonte conforme explicitado no capítulo 3.

Adicionalmente, serão consideradas apenas técnicas de classificação por assunto limitada a nível de área do conhecimento (ex: computação/cinema/medicina/direito), sem levar em consideração as diferenças entre matérias (ex: linguagens formais/inteligência artificial/mineração de dados).

\section{Abordagem}
A abordagem escolhida por este projeto foi a de obter uma base de dados de questões e tratá-la da maneira mais conveniente de modo a se adequar aos seus objetivos. Tal obtenção e tratamento são abordadas no capítulo 3.

Com a base de dados, a próxima etapa é a de utilizar as principais técnicas de classificação de textos genéricos, descritas no capítulo 4, na classificação do assunto de suas questões. Nessa etapa, cada um dos métodos será otimizado para essa base de dados, validando suas arquiteturas e seus hiperparâmetros. Com esses resultados, é possível comparar a performance desses métodos.

Com base nessa comparação, avalia-se a viabilidade de implementar um modelo unificado dos modelos que utilizará os resultados anteriores e determinará qual é a melhor previsão de assunto da questão (Ensemble learning). O maior detalhamento desses modelos está no capítulo 5.

\section{Metodologia}
\label{sec:metodology}
O trabalho será dividido em seis etapas conforme a figura \ref{fig:fluxogram}.

\begin{figure}[!ht]
  \centering
  \tikzstyle{block} = [rectangle, draw, text width=10em, text centered, rounded      corners, minimum height=3em]
  \begin{tikzpicture}
   [node distance=2cm,
   start chain=going below,]
    \node (n1) at (0,0) [block]  {Crawler};
    \node (n2) [block, below of=n1] {Tratamento};
    \node (n3) [block, below of=n2] {Implementação};
    \node (n4) [block, below of=n3] {Análise};
    \node (n5) [block, below of=n4] {Desenvolvimento};
    \node (n6) [block, below of=n5] {Avaliação};

    \draw [->] (n1) -- (n2);
    \draw [->] (n2) -- (n3);
    \draw [->] (n3) -- (n4);
    \draw [->] (n4) -- (n5);
    \draw [->] (n5) -- (n6);
    %\draw [->] (n4.east) -| ++(1,0) |- (n3.east);
    %\draw [->] (n4.west) -| ++(-1,0) |- (n2.west);
  \end{tikzpicture}
  \caption{Fluxograma de trabalho}
  \label{fig:fluxogram}
\end{figure}
\subsection{Crawler}
A primeira etapa do trabalho corresponde à obtenção de dados para serem utilizados nos algoritmos de aprendizado de máquina. Isso é feito a partir de Web scraping de questões de concursos públicos já rotuladas com os seus respectivos assuntos.
\subsection{Tratamento}
Em seguida, há uma fase de processamento de dados que remove inconsistências e prepara uma interface ideal para os algoritmos que serão aplicados posteriormente.
\subsection{Implementação}
A terceira fase consiste na implementação de alguns modelos consagrados de classificação genérica de texto no conjunto de dados em questão.
\subsection{Análise}
A seguir, será feita uma análise de desempenho para cada algoritmo que poderá resultar na variação de seus hiperparâmetros implementados e tratamento do conjunto de dados.
\subsection{Desenvolvimento}
Da análise obtida na etapa anterior será desenvolvida uma técnica híbrida especializada em questões.
\subsection{Avaliação}
Por fim, os resultados obtidos serão avaliados e contrastados com os modelos reais para aferição de alcance dos objetivos.

\section{Organização}
%do documento
Sobre os capítulos subsequentes a seguinte distribuição de conteúdo:
\begin{itemize}
\item Capítulo 2 - Trabalhos relacionados:
exposição de trabalhos relacionados ao tema de classificação de texto;
\item Capítulo 3 - Obtenção da base de dados:
metodologia utilizada para obtenção e processamento da base de dados utilizada como referência no projeto;
\item Capítulo 4 - Fundamentação Teórica:
descrição dos principais conceitos utilizados em cada um dos algoritmos de classificação que serão implementados;
\item Capítulo 5 - Algoritmos de classificação:
descrição, metodologia de implementação e resultados de cada um dos algoritmos implementados;
\item Capítulo 6 - Comparação e conclusão:
comparação dos resultados dos entre diferentes modelos e algoritmos implementados no capítulo anterior.
\end{itemize}