\chapter{Desenvolvimento}
\noindent
\section{Web Scrapping}
% Rota dos concursos
\section{Pré-Processamento}
%De modo a ampliar a eficiência da classificação, este trabalho separa em dois níveis de classificação para o assunto das questões. O primeiro nível diz respeito a disciplina da questão, baseada nas disciplinas do edital do concurso, como, por exemplo, história, geografia e matemática. O segundo nível diz respeito ao assunto da questão propriamente dito, que pode ser entendido como um subtópico da disciplina da questão. Esse subtópico é, por exemplo, Revolução Francesa, que seria um subtópico de história, ou topografia, que é um subtópico de geografia.
%Essa divisão em dois níveis funciona para resolver ambiguidades de palavras chaves e seus pesos dentro do texto da questão. Por exemplo, uma questão de "cinética física" poderia ser levada para o tópico "cinética química" pois a relevância da palavra cinética é maior neste tópico.

\section{Algoritmos implementados}
\subsection{Bag of Words}
\subsection{Redes Neural}
\subsection{Redes Neural Convolucional}
\subsection{Redes Neural Recorrente}